\section{仪器设备、研究经费}

\subsection{已有基础}

\subsubsection{知识与技能储备}

\begin{itemize}
    \item \textbf{理论基础}:深度学习、计算机视觉、语义分割等核心知识
    \item \textbf{编程能力}:Python、PyTorch框架,复现过多个经典模型
    \item \textbf{论文阅读}:系统阅读CVPR/ICCV等顶会论文60+篇
\end{itemize}

\subsubsection{计算资源}

\textbf{主要设备}(个人/实验室):
\begin{itemize}
    \item GPU服务器:4×A100 40GB(实验室共享)
    \item 个人工作站:RTX 3090 24GB
    \item 云计算资源:按需租用(预算1.2万元)
\end{itemize}

\textbf{软件环境}:PyTorch 2.0、CUDA 11.8、GDAL、QGIS等已配置完成

\subsubsection{数据采集条件}

\begin{itemize}
    \item \textbf{无人机}:与农学院合作,共享DJI Phantom 4 RTK
    \item \textbf{试验田}:已联系XX县YY镇3块水稻田(约30亩)
    \item \textbf{标注工具}:LabelMe、QGIS等开源工具
\end{itemize}

\subsection{经费预算}

\subsubsection{总预算:约8万元(个人/导师承担)}

\begin{table}[h]
\centering
\caption{经费预算明细}
\begin{tabular}{lrr}
\toprule
类别 & 预算(元) & 说明 \\
\midrule
数据采集 & 30,000 & 无人机使用、差旅、农田协调 \\
数据标注 & 15,000 & 外包标注1500张×5元+自标 \\
计算资源 & 15,000 & 云GPU租用(高峰期补充) \\
论文发表 & 10,000 & 版面费、会议注册 \\
其他支出 & 10,000 & 存储、网络、杂费 \\
\midrule
\textbf{合计} & \textbf{80,000} & \\
\bottomrule
\end{tabular}
\end{table}

\textbf{经费来源}:
\begin{itemize}
    \item 导师科研经费支持(主要)
    \item 个人承担部分(数据标注、云计算)
    \item 实验室公共资源(GPU服务器、软件环境)
\end{itemize}

\subsection{工作量估算}

\begin{table}[h]
\centering
\caption{主要工作量}
\begin{tabular}{lcc}
\toprule
任务 & 数量/时长 & 说明 \\
\midrule
数据采集 & 3000张 & 3块田地,多时相 \\
数据标注 & 400小时 & 外包+自标 \\
算法开发 & 6个月 & SACL、MDA-CMC、系统 \\
论文撰写 & 5个月 & 中英文各1篇+学位论文 \\
\bottomrule
\end{tabular}
\end{table}

\subsection{可行性分析}

\subsubsection{技术可行性}

\begin{itemize}
    \item DFormerv2代码开源,可作为技术基础
    \item 外观不变性学习理论研究充分
    \item 创新方向明确:SACL和MDA-CMC针对农业遥感的独特挑战
\end{itemize}

\subsubsection{资源保障}

\begin{itemize}
    \item \textbf{计算资源}:实验室GPU服务器+云计算补充,完全满足需求
    \item \textbf{数据获取}:已与农学院、XX县农业部门建立合作关系
    \item \textbf{经费充足}:导师科研经费支持,8万元预算合理
    \item \textbf{时间合理}:14个月工期(2025.03-2026.05),节奏紧凑但可行
\end{itemize}

\subsubsection{风险与应对}

\begin{table}[h]
\centering
\caption{主要风险及应对措施}
\begin{tabular}{p{4cm}p{5cm}p{4cm}}
\toprule
风险 & 影响 & 应对措施 \\
\midrule
天气影响数据采集 & 延误1-2个月 & 预留缓冲时间,多时相采集 \\
算法效果不达预期 & 论文投稿困难 & 多种基线对比,充分消融实验 \\
GPU资源不足 & 实验周期延长 & 云计算补充,优化代码效率 \\
\bottomrule
\end{tabular}
\end{table}

\subsection{预期成果}

\subsubsection{学术成果}

\begin{itemize}
    \item \textbf{学位论文}:1篇(必须)
    \item \textbf{期刊论文}:中文核心1篇(目标:农业工程学报)
    \item \textbf{会议论文}:英文会议1篇(目标:CVPR Workshop或农业领域会议)
\end{itemize}

\subsubsection{技术成果}

\begin{itemize}
    \item \textbf{算法模型}:SACL和MDA-CMC开源代码
    \item \textbf{数据集}:水稻倒伏RGB-D数据集(计划公开)
    \item \textbf{系统原型}:水稻倒伏监测GIS系统(演示版)
\end{itemize}

\subsubsection{应用价值}

\begin{itemize}
    \item 为农业保险提供倒伏损失快速评估工具
    \item 为农业部门提供灾情监测技术支持
    \item 方法可扩展到其他作物(小麦、玉米等)和其他遥感任务
\end{itemize}

\subsection{总结}

本研究具备充分的可行性保障:

\begin{itemize}
    \item \textbf{创新性}:SACL和MDA-CMC针对农业遥感原创设计,理论与应用并重
    \item \textbf{资源充足}:计算资源、数据获取、经费支持均已落实
    \item \textbf{风险可控}:主要风险已识别,应对措施明确
    \item \textbf{成果明确}:学术论文、技术成果、应用价值清晰
\end{itemize}

预期在14个月内(2025.03-2026.05)顺利完成研究任务,取得创新性成果。
