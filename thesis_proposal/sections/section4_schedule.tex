\section{论文工作的总体安排及进度}

\subsection{总体时间规划(14个月)}

\begin{table}[h]
\centering
\caption{研究工作总体进度安排(2025.03-2026.05)}
\begin{tabular}{clp{6cm}p{4cm}}
\toprule
阶段 & 时间 & 主要任务 & 里程碑成果 \\
\midrule
一 & 1-2月 & 文献调研、环境搭建 & 实验环境ready \\
   & (2025.03-04) & 数据采集准备 & 数据采集开始 \\
\midrule
二 & 3-4月 & 数据采集与标注 & 完成3000张标注 \\
   & (2025.05-06) & DFormerv2基线训练 & mIoU≥85\% \\
\midrule
三 & 5-7月 & SACL方法设计与实现 & 跨田块提升 \\
   & (2025.07-09) & 跨田块泛化实验 & +6\%以上 \\
\midrule
四 & 8-10月 & MDA-CMC框架研究 & 完整方法实现 \\
   & (2025.10-12) & 全面对比实验 & 消融实验完成 \\
\midrule
五 & 11-12月 & GIS系统开发 & 系统原型 \\
   & (2026.01-02) & 论文撰写投稿 & 投稿完成 \\
\midrule
六 & 13-14月 & 学位论文撰写 & 学位论文完成 \\
   & (2026.03-05) & 毕业答辩准备 & 答辩通过 \\
\bottomrule
\end{tabular}
\end{table}

\subsection{关键里程碑}

\subsubsection{第一阶段:前期准备(1-2月)}

\textbf{主要任务}:
\begin{itemize}
    \item 深度文献调研:DFormerv2、外观不变性学习、农业遥感等
    \item 开题报告撰写与答辩
    \item 实验环境搭建:PyTorch、CUDA、DFormerv2代码复现
    \item 数据采集准备:联系农学院、确定试验田、设备调试
\end{itemize}

\textbf{里程碑}:开题答辩通过,环境ready

\subsubsection{第二阶段:数据与基线(3-4月)}

\textbf{主要任务}:
\begin{itemize}
    \item 数据采集:3块田地共采集3000张RGB-D图像对
    \item 数据标注:外包+自标完成3000张
    \item DFormerv2训练:在我们数据集上训练基线模型
    \item 性能评估:测试集mIoU、边界IoU等指标
\end{itemize}

\textbf{里程碑}:数据集完成,基线模型mIoU≥85\%

\subsubsection{第三阶段:SACL方法(5-7月)}

\textbf{主要任务}:
\begin{itemize}
    \item SACL设计:自适应权重模块、深度质量评估、空间编码
    \item 双层优化:几何引导+特征对齐+梯度一致性
    \item 跨田块实验:单域训练、多域测试,验证泛化能力
    \item 消融实验:各组件贡献度分析
\end{itemize}

\textbf{里程碑}:跨田块平均mIoU提升≥6\%(vs 基线)

\subsubsection{第四阶段:MDA-CMC与扩展(8-10月)}

\textbf{主要任务}:
\begin{itemize}
    \item MDA-CMC设计:域内一致性+域间对比学习
    \item 半监督学习:伪标签生成、跨模态验证、迭代训练
    \item 消融实验:各组件贡献度分析
    \item 对比实验:vs DFormerv2、CMX、TransUNet等SOTA方法
\end{itemize}

\textbf{里程碑}:完整方法体系,少样本适应(10\%数据→90\%性能)

\subsubsection{第五阶段:系统与论文(11-12月)}

\textbf{主要任务}:
\begin{itemize}
    \item GIS系统开发:WebGIS前端+后端API+模型部署
    \item 论文撰写投稿:中文核心期刊或英文会议
    \item 系统测试:真实场景验证、性能优化
\end{itemize}

\textbf{里程碑}:GIS系统原型、论文投稿完成

\subsubsection{第六阶段:学位论文与答辩(13-14月)}

\textbf{主要任务}:
\begin{itemize}
    \item 学位论文撰写:≥5万字,包含所有工作内容
    \item 论文修改完善:根据导师意见修改
    \item 毕业答辩准备:PPT、演讲稿、答辩预演
\end{itemize}

\textbf{里程碑}:学位论文完成,答辩通过,顺利毕业

\subsection{关键时间节点}

\begin{table}[h]
\centering
\caption{关键时间节点(2025.03-2026.05)}
\begin{tabular}{cp{10cm}}
\toprule
时间 & 关键节点 \\
\midrule
2025.04 & 实验环境搭建完成,数据采集开始 \\
2025.06 & 数据集完成(3000张),DFormerv2基线模型(mIoU≥85\%) \\
2025.09 & SACL方法完成,跨田块实验验证(提升≥6\%) \\
2025.12 & MDA-CMC完整方法体系完成,消融实验完成 \\
2026.02 & GIS系统原型完成,论文投稿 \\
2026.04 & 学位论文定稿 \\
2026.05 & 毕业答辩,顺利毕业 \\
\bottomrule
\end{tabular}
\end{table}

\subsection{风险与应对}

\subsubsection{进度风险}

\begin{itemize}
    \item \textbf{风险}:天气影响数据采集进度
    \item \textbf{应对}:预留1-2个月缓冲时间,雨季前完成主要采集
\end{itemize}

\begin{itemize}
    \item \textbf{风险}:算法效果不达预期,需要反复调试
    \item \textbf{应对}:多个基线对比,充分消融实验,降低单一方法依赖
\end{itemize}

\subsubsection{论文发表风险}

\begin{itemize}
    \item \textbf{风险}:期刊/会议拒稿
    \item \textbf{应对}:准备备选期刊清单,根据审稿意见及时修改重投
\end{itemize}

\subsection{进度保障措施}

\begin{itemize}
    \item \textbf{周报制度}:每周向导师汇报进展,及时发现问题
    \item \textbf{月度审查}:每月检查进度,对比计划,调整节奏
    \item \textbf{并行工作}:数据采集与算法开发并行,提高效率
    \item \textbf{备选方案}:关键环节准备Plan B,降低风险
\end{itemize}

\subsection{总结}

本研究14个月的时间安排科学合理(2025.03-2026.05):

\begin{itemize}
    \item \textbf{阶段清晰}:6个阶段,任务明确,衔接紧密
    \item \textbf{节奏紧凑}:前期准备2个月,数据+基线2个月,方法研究6个月,系统+论文4个月
    \item \textbf{里程碑明确}:每阶段有可量化的成果指标
    \item \textbf{风险可控}:关键环节预留缓冲,备选方案充分
\end{itemize}

在导师指导下,严格按照计划执行,有信心按时完成高质量的硕士学位论文。
