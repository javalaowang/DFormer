\section{研究方法及技术路线}

\subsection{研究方法}

\subsubsection{数据采集与预处理方法}

\textbf{(1)遥感数据获取}

采用无人机低空遥感技术,具体方案:

\begin{itemize}
    \item \textbf{RGB影像}:DJI Phantom 4 Pro相机(20MP,2-5 cm/pixel)
    \item \textbf{深度数据}:LiDAR点云(Velodyne VLP-16或DJI L1)生成DSM
    \item \textbf{飞行参数}:
    \begin{itemize}
        \item 飞行高度:50-100米
        \item 重叠率:前向80\%,旁向70\%
        \item 飞行速度:5-8 m/s
    \end{itemize}
    \item \textbf{地面真值}:GPS定位倒伏区域,测量倒伏角度和面积
\end{itemize}

\textbf{(2)深度图生成}

\begin{enumerate}
    \item \textbf{LiDAR点云处理}:
    \begin{itemize}
        \item 点云滤波:去除噪声点
        \item 地面分类:Cloth Simulation Filter(CSF)算法
        \item DSM生成:IDW插值(分辨率与RGB一致)
    \end{itemize}
    
    \item \textbf{相对高度计算}:
    \begin{equation}
    H_{relative}(x,y) = H_{DSM}(x,y) - H_{ground}(x,y)
    \end{equation}
    
    \item \textbf{深度归一化}:
    \begin{equation}
    D_{norm}(x,y) = \frac{H_{relative}(x,y) - H_{min}}{H_{max} - H_{min}} \times 255
    \end{equation}
\end{enumerate}

\textbf{(3)配准与质量控制}

\begin{itemize}
    \item RGB-DSM配准:基于GPS/IMU数据和特征点匹配
    \item 配准精度:<1像素误差
    \item 质量筛选:去除深度缺失>20\%的图像
\end{itemize}

\subsubsection{模型训练方法}

\textbf{(1)训练配置}

\begin{table}[h]
\centering
\caption{模型训练超参数}
\begin{tabular}{ll}
\toprule
参数 & 值 \\
\midrule
优化器 & AdamW \\
初始学习率 & 6e-5 \\
学习率策略 & Poly(power=0.9) \\
权重衰减 & 0.01 \\
Batch size & 16(4×4 GPU) \\
总Epochs & 300 \\
Warmup epochs & 10 \\
\midrule
$\lambda_1$(域内一致性) & 0.5 \\
$\lambda_2$(域间一致性) & 0.3 \\
$\beta$(梯度一致性) & 0.2 \\
$\tau$(对比学习温度) & 0.07 \\
\bottomrule
\end{tabular}
\end{table}

\textbf{(2)数据增强策略}

标准增强 + 外观增强:
\begin{itemize}
    \item 几何增强:RandomFlip、RandomRotate(90°倍数)、RandomScale(0.5-2.0)
    \item RandomCrop:512×512
    \item \textbf{外观增强}(模拟跨田块差异):
    \begin{itemize}
        \item ColorJitter(brightness=0.4, contrast=0.4, saturation=0.4)模拟不同生长状态的叶色差异
        \item 添加田面反射效果(模拟不同背景条件)
        \item Cutout:随机遮挡(模拟不同冠层密度和阴影)
    \end{itemize}
\end{itemize}

\textbf{(3)损失函数}

\begin{equation}
\begin{split}
\mathcal{L}_{total} = & \mathcal{L}_{CE}(y, \hat{y}) + \lambda_{dice} \mathcal{L}_{Dice}(y, \hat{y}) \\
& + \lambda_{aux} \mathcal{L}_{aux} \\
& + \lambda_1 \mathcal{L}_{intra} + \lambda_2 \mathcal{L}_{inter}
\end{split}
\end{equation}

其中:
\begin{itemize}
    \item $\mathcal{L}_{CE}$:交叉熵损失
    \item $\mathcal{L}_{Dice}$:Dice损失(处理类别不平衡)
    \item $\mathcal{L}_{aux}$:中间层辅助损失
    \item $\mathcal{L}_{intra}, \mathcal{L}_{inter}$:一致性损失
\end{itemize}

\subsubsection{评估方法}

\textbf{(1)定量评估}

\begin{itemize}
    \item mIoU(主要指标)
    \item Pixel Accuracy
    \item 倒伏类IoU(关键类别)
    \item F1-Score
    \item Kappa系数
\end{itemize}

\textbf{(2)定性评估}

\begin{itemize}
    \item 可视化对比(预测 vs 真值)
    \item 错误案例分析
    \item 边界质量评估
\end{itemize}

\textbf{(3)效率评估}

\begin{itemize}
    \item 参数量(Params)
    \item 计算量(FLOPs)
    \item 推理速度(FPS、ms/image)
    \item GPU显存占用
\end{itemize}

\subsection{技术路线}

\begin{figure}[h]
\centering
\begin{minipage}{0.9\textwidth}
\small
\textbf{阶段一:文献调研与数据准备(第1-2个月)}
\begin{itemize}
    \item 深入研读DFormerv2、v-CLR等核心论文
    \item 调研农业遥感和倒伏监测文献(≥50篇)
    \item 设计数据采集方案,联系合作单位
    \item 搭建实验环境(GPU服务器、开发环境)
    \item \textbf{里程碑}:开题答辩通过
\end{itemize}

\textbf{阶段二:基线模型构建(第3-4个月)}
\begin{itemize}
    \item 开始数据采集(地块A,≥1000张)
    \item 数据预处理pipeline开发
    \item 数据标注(700张训练+200验证+100测试)
    \item DFormerv2代码复现,在NYU Depth v2验证
    \item 模型适配到水稻倒伏场景
    \item 基线训练和性能评估
    \item \textbf{里程碑}:基线mIoU≥80\%
\end{itemize}

\textbf{阶段三:SACL方法研究(第5-7个月)}
\begin{itemize}
    \item 继续数据采集(地块B、C,各≥1000张)
    \item SACL理论分析和公式推导
    \item 空间自适应权重模块实现
    \item 双层一致性框架开发
    \item 超参数网格搜索($\lambda_1, \lambda_2, \beta$)
    \item 消融实验:验证各组件贡献
    \item \textbf{里程碑}:跨地块提升≥8\%
\end{itemize}

\textbf{阶段四:MDA-CMC与半监督学习(第8-10个月)}
\begin{itemize}
    \item 数据采集完成(地块D、E,≥2000张)
    \item 多域对齐框架实现
    \item 对比学习模块开发
    \item 半监督学习方法实现
    \item Few-shot适应实验
    \item 完整对比实验
    \item \textbf{里程碑}:半监督达到全监督90\%
\end{itemize}

\textbf{阶段五:GIS系统开发(第11-13个月)}
\begin{itemize}
    \item 系统架构设计
    \item 前端开发(地图组件、可视化)
    \item 后端开发(API、数据库)
    \item 模型优化部署(TensorRT、量化)
    \item 系统集成测试
    \item 用户测试和反馈
    \item \textbf{里程碑}:可演示的系统原型
\end{itemize}

\textbf{阶段六:论文撰写(第14-18个月)}
\begin{itemize}
    \item 补充实验和数据整理
    \item 论文各章节撰写
    \item 英文论文撰写和投稿(可选)
    \item 硕士论文修改和完善
    \item 预答辩和正式答辩
    \item \textbf{里程碑}:答辩通过,顺利毕业
\end{itemize}
\end{minipage}
\end{figure}

\subsection{可能遇到的问题及解决方案}

\subsubsection{数据采集相关}

\textbf{问题1:天气影响数据采集}

\begin{itemize}
    \item \textbf{风险}:台风、暴雨导致倒伏,但无法飞行采集
    \item \textbf{解决方案}:
    \begin{itemize}
        \item 预留缓冲时间,在天气窗口密集采集
        \item 多地块采集,降低单点风险
        \item 预先准备应急方案(地面摄影补充)
    \end{itemize}
\end{itemize}

\textbf{问题2:倒伏样本不足}

\begin{itemize}
    \item \textbf{风险}:某些年份、地块倒伏发生少
    \item \textbf{解决方案}:
    \begin{itemize}
        \item 扩大采集范围,5个地块提高覆盖率
        \item 利用历史遥感数据,补充标注
        \item 与农科院合作,在人工倒伏试验田采集
        \item 数据增强生成更多变化样本
    \end{itemize}
\end{itemize}

\textbf{问题3:深度数据质量差}

\begin{itemize}
    \item \textbf{风险}:LiDAR点云稀疏,DSM有空洞
    \item \textbf{解决方案}:
    \begin{itemize}
        \item 降低飞行高度,提高点云密度
        \item 使用插值算法填补空洞(Kriging、IDW)
        \item SACL的自适应权重会降低噪声影响
        \item 备选:SfM(Structure from Motion)重建深度
    \end{itemize}
\end{itemize}

\subsubsection{模型训练相关}

\textbf{问题4:跨域泛化效果不达预期}

\begin{itemize}
    \item \textbf{可能原因}:
    \begin{itemize}
        \item 训练数据分布单一
        \item 一致性约束权重不合适
        \item 域间差异过大
    \end{itemize}
    \item \textbf{解决方案}:
    \begin{itemize}
        \item 增加训练数据多样性(多地块联合训练)
        \item 网格搜索最优$\lambda_1, \lambda_2$
        \item 引入域自适应技术(DANN、ADDA)
        \item 多任务学习(同时学习倒伏检测和作物分类)
        \item 课程学习:从简单域到困难域渐进训练
    \end{itemize}
\end{itemize}

\textbf{问题5:训练不收敛或不稳定}

\begin{itemize}
    \item \textbf{现象}:Loss震荡、梯度爆炸/消失
    \item \textbf{解决方案}:
    \begin{itemize}
        \item 梯度裁剪(clip\_norm=1.0)
        \item 降低学习率或使用更保守的scheduler
        \item 减小一致性损失权重
        \item 分阶段训练:先训练基线,再加一致性约束
        \item 使用混合精度训练(AMP)提高稳定性
    \end{itemize}
\end{itemize}

\textbf{问题6:类别不平衡}

\begin{itemize}
    \item \textbf{问题}:倒伏像素<20\%,正常像素>70\%
    \item \textbf{解决方案}:
    \begin{itemize}
        \item Focal Loss:$FL(p_t) = -(1-p_t)^\gamma \log(p_t)$,$\gamma=2$
        \item 类别权重:$w_{lodging}=3.0, w_{normal}=1.0, w_{bg}=0.5$
        \item 难例挖掘:训练时重采样困难样本
        \item Dice Loss:对类别不平衡鲁棒
    \end{itemize}
\end{itemize}

\subsubsection{系统开发相关}

\textbf{问题7:大规模遥感图像处理}

\begin{itemize}
    \item \textbf{挑战}:单张图像可达10000×10000像素,GPU显存不足
    \item \textbf{解决方案}:
    \begin{itemize}
        \item 滑动窗口切分:512×512瓦片,overlap=64
        \item 批量推理:多GPU并行
        \item 模型量化:INT8量化,减少显存50\%
        \item TensorRT加速:推理速度提升2-3倍
    \end{itemize}
\end{itemize}

\textbf{问题8:实时性要求}

\begin{itemize}
    \item \textbf{目标}:<10秒/km²(约1000张瓦片)
    \item \textbf{解决方案}:
    \begin{itemize}
        \item 使用轻量模型DFormerv2-S(参数量减半)
        \item GPU推理优化(TensorRT、ONNX Runtime)
        \item 异步处理:后台任务队列(Celery)
        \item 结果缓存:已处理区域不重复计算
    \end{itemize}
\end{itemize}

\textbf{问题9:GIS系统开发复杂度高}

\begin{itemize}
    \item \textbf{挑战}:前后端开发、地理信息处理、大数据管理
    \item \textbf{解决方案}:
    \begin{itemize}
        \item 采用成熟开源框架(Leaflet、Flask、PostGIS)
        \item MVP策略:核心功能优先,逐步完善
        \item 寻求合作:与计算机专业同学合作开发
        \item 云服务:使用阿里云/腾讯云,降低运维复杂度
    \end{itemize}
\end{itemize}

\subsection{技术路线图}

\begin{figure}[h]
\centering
\fbox{
\begin{minipage}{0.85\textwidth}
\centering
\textbf{技术路线总览}

\vspace{0.3cm}

\begin{tabular}{c}
\framebox[0.8\textwidth]{文献调研 + 理论基础} \\
$\downarrow$ \\
\framebox[0.8\textwidth]{数据采集 + 标注} \\
$\downarrow$ \\
\framebox[0.8\textwidth]{DFormerv2基线模型} \\
$\downarrow$ \\
\framebox[0.8\textwidth]{SACL:空间自适应一致性学习} \\
$\downarrow$ \\
\framebox[0.8\textwidth]{MDA-CMC:多域对齐框架} \\
$\downarrow$ \\
\framebox[0.8\textwidth]{扩展:半监督学习} \\
$\downarrow$ \\
\framebox[0.8\textwidth]{GIS系统开发与集成} \\
$\downarrow$ \\
\framebox[0.8\textwidth]{论文撰写与答辩} \\
\end{tabular}
\end{minipage}
}
\caption{技术路线流程图}
\end{figure}

\subsection{关键技术细节}

\subsubsection{自适应权重学习细节}

\textbf{网络结构}:
\begin{verbatim}
MLP_weight(
    Input: [q_d, p_spatial, c_scene]  # 维度: 3
    FC1: 3 → 64, ReLU
    FC2: 64 → 32, ReLU
    FC3: 32 → 1, Sigmoid
    Output: w ∈ [0, 1]
)
\end{verbatim}

\textbf{训练策略}:
\begin{itemize}
    \item MLP参数与主网络联合训练
    \item 初始化:$w=0.5$(均匀权重)
    \item 正则化:$\mathcal{L}_{reg} = 0.01 \cdot \|w - 0.5\|_2^2$(避免退化)
\end{itemize}

\subsubsection{多尺度一致性实现}

在4个特征层($L_1, L_2, L_3, L_4$)应用约束:

\begin{itemize}
    \item $L_1$:分辨率1/4,通道64,权重0.5
    \item $L_2$:分辨率1/8,通道128,权重0.75
    \item $L_3$:分辨率1/16,通道256,权重1.0
    \item $L_4$:分辨率1/32,通道512,权重1.0
\end{itemize}

\textbf{原理}:高层特征更抽象,更接近语义,权重更大。

